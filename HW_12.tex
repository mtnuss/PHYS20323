%%%%%%%%%%%%%%%%%%%%%%%%%%%%%%%%%%%%%%%%%%%%%%%%%%%%%%%%%%%%
%%%%%%%%%%%%%%%%%%%%%%%%%%%%%%%%%%%%%%%%%%%%%%%%%%%%%%%%%%%%
%%%%%%%%%%%%%%%%%%%%%%%%%%%%%%%%%%%%%%%%%%%%%%%%%%%%%%%%%%%%
%%%%%%%%%%%%%%%%%%%%%%%%%%%%%%%%%%%%%%%%%%%%%%%%%%%%%%%%%%%%
%%%%%%%%%%%%%%%%%%%%%%%%%%%%%%%%%%%%%%%%%%%%%%%%%%%%%%%%%%%%
\documentclass[12pt]{article}
\usepackage{epsfig}
\usepackage{times}
\usepackage{amsmath}
\renewcommand{\topfraction}{1.0}
\renewcommand{\bottomfraction}{1.0}
\renewcommand{\textfraction}{0.0}
\setlength {\textwidth}{6.6in}
\hoffset=-1.0in
\oddsidemargin=1.00in
\marginparsep=0.0in
\marginparwidth=0.0in                                                                               
\setlength {\textheight}{9.0in}
\voffset=-1.00in
\topmargin=1.0in
\headheight=0.0in
\headsep=0.00in
\footskip=0.50in                                         
\setcounter{page}{1}
\begin{document}
\def\pos{\medskip\quad}
\def\subpos{\smallskip \qquad}
\newfont{\nice}{cmr12 scaled 1250}
\newfont{\name}{cmr12 scaled 1080}
\newfont{\swell}{cmbx12 scaled 800}
%%%%%%%%%%%%%%%%%%%%%%%%%%%%%%%%%%%%%%%%%%%%%%%%%%%%%%%%%%%%
%     DO NOT CHANGE ANYTHING ABOVE THIS LINE
%%%%%%%%%%%%%%%%%%%%%%%%%%%%%%%%%%%%%%%%%%%%%%%%%%%%%%%%%%%%
%     DO NOT CHANGE ANYTHING ABOVE THIS LINE
%%%%%%%%%%%%%%%%%%%%%%%%%%%%%%%%%%%%%%%%%%%%%%%%%%%%%%%%%%%%
%     DO NOT CHANGE ANYTHING ABOVE THIS LINE
%%%%%%%%%%%%%%%%%%%%%%%%%%%%%%%%%%%%%%%%%%%%%%%%%%%%%%%%%%%%

\begin{center}
{\large\bf{
PHYSI 20323/60323: Fall 2020 - LaTeX Example}}\\\vskip0.25in
%%%%%%%%%%%%%%%%%%%%%%%%%%%%%%%%%%%%%%%%%%%%%%%%%%%%%%%%%%%%
%%%%%%%%%%%%%%%%%%%%%%%%%%%%%%%%%%%%%%%%%%%%%%%%%%%%%%%%%%%%
\end{center}

%%%%%%%%%%%%%%%%%%%%%%%%%%%%%%%%%%%%%%%%%%%%%%%%%%%%%%%%%%%%
% Section Heading
%%%%%%%%%%%%%%%%%%%%%%%%%%%%%%%%%%%%%%%%%%%%%%%%%%%%%%%%%%%%
\vskip0.1in



%%%%%%%%%%%%%%%%%%%%%%%%%%%%%%%%%%%%%%%%%%%%%%%%%%%%%%%%%%%%
% Bullet Point & Numbered list - lists can also be nested as below
%%%%%%%%%%%%%%%%%%%%%%%%%%%%%%%%%%%%%%%%%%%%%%%%%%%%%%%%%%%%
\begin{enumerate}
\item Consider a particle confined in a two-dimensional infinite square well
\begin{equation*}
	V(x,y)=
		\begin{cases}
			0, & \text{if 0 $\leq$ x $\leq$ a,  0$<$y$<$a}\\
			\infty , & \text{otherwise}
		\end{cases}
\end{equation*}
The eigenfunctions have the form:
\begin{equation*}
	\Psi (x,y) = 
		\frac{2}{a} \sin \left( \frac{n \pi x}{a} \right) \sin \left( \frac{m \pi y}{a} \right)
\end{equation*}
with the corresponding energies being given by:
\begin{equation*}
	E_{nm} =
		\left( n^2 + m^2 \right) \frac{\pi ^2 \hbar ^2}{2ma^2}
\end{equation*}
(a) 5 points) What are the levels of degeneracy of the five lowest energy values? \\
(b) (5 points) Consider a perturbation given by:
\begin{equation*}
	\hat{H}^\prime = 
		a^2 V_{o} \delta \left( x - \frac{a}{2} \right) \delta \left( y - \frac{a}{2} \right)
\end{equation*}
\-   \ \-   \ \-   \ Calculate the first order correction to the ground state energy.

\item \textbf{The following question refers to stars in the Table below.} \\
Note: There may be multiple answers.

\begin{tabular}{|l|c|c|c|c|c|}
\hline
Name & Mass & Luminosity & Lifetime & Temperature & Radius \\
\hline
Zeta & 60. $M_{sun}$ & $10^6$ $L_{sun}$ & $8.0 \times 10^5$ years & & \\
\hline
Epsilon & 6.0 $M_{sun}$ & $10^3$ $L_{sun}$ & & 20,000 K & \\
\hline 
Delta & 2.0 $M_{sun}$ & & $5.0 \times 10^8$ years & & 2 $R_{sun}$ \\
\hline
Beta & 1.3 $M_{sun}$ & 3.5 $L_{sun}$ & & & \\
\hline
Alpha & 1.0 $M_{sun}$ & & & & 1 $R_{sun}$ \\
\hline
Gamma & 0.7 $M_{sun}$ & & $4.5 \times 10^10$ years & 5000 K & \\
\hline
\end{tabular}

(a) (4 points) Which of these stars will produce a planetary nebula at the end of their life. \vskip0.25in

(b) (4 points) Elements heavier than \textit{Carbon} will be produced in which stars.
\end{enumerate}


%%%%%%%%%%%%%%%%%%%%%%%%%%%%%%%%%%%%%%%%%%%%%%%%%%%%%%%%%%%%



\end{document}
